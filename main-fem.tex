\documentclass[11pt,a4paper,sans]{moderncv}

% style options are 'casual' (default), 'classic', 'oldstyle' and 'banking'
\moderncvstyle{banking}
% color options 'purple' (default), 'orange', 'green', 'red', 'purple', 'grey' and  'black'
\moderncvcolor{purple}

% character encoding
\usepackage[utf8]{inputenc}

% adjust the page margins
\usepackage[scale=0.8]{geometry}

% personal data
\name{GISELA}{MORINIGO}
\address{Monroe 3772 1F}{Ciudad de Buenos Aires (CP 1430)}{Bs. As., Argentina}
\phone[mobile]{54-11-5972-0764}
\email{gisemo@gmail.com}
\extrainfo{Passport 29506398N, US VISA B1}

%----------------------------------------------------------------------------------
%            content
%----------------------------------------------------------------------------------
\begin{document}

%-----       resume       ---------------------------------------------------------
\makecvtitle

\section{Education}

\cventry{2020--Present}{Social and Cultural Anthropology}{Universidad Nacional de San Martín, UNSAM}{Degree}{}{}
\cventry{2012--2014}{Software Engineering Specialist}{Universidad Católica de Buenos Aires, FCFI}{Post-Degree}{}{}
\cventry{2000--2006}{System Engineering}{Universidad Tecnológica Nacional, FRBA}{Degree}{}{}


\section{Personal objectives}

I have been working in the software development industry for several years and in the last 5 years I have participated actively in the feminist and social movement in Argentina.
I have been so deeply involved in the feminism that I joined a local organization Ahora Que Si Nos Ven that fights against femicides, we generate analysis and monthly reports. I fully participated on "Marea Verde", the local movement that mobilized to legalize/decriminalize abortion here in Argentina.
I want to start applying my experience and knowledge in project management, communication and interpersonal skills in a feminist organization.
I enjoy working with teams from different cultures and in diverse countries. I am a very analytical person with good communication and problem solving skills.
I like dynamic environments and I adapt easily to changes.



\section{Languages}

\cventry{}{}{Spanish}{Native}{}{}
\cventry{}{Working closely with English-speaking stakeholders.}{English}{Fluent}{}{\break}
\cventry{}{Worked with french-speaking team members.}{French}{Intermediate}{}{\break}
\cventry{}{}{Italian}{Basic}{}{}

\section{Volunteer Activities and Interests}
\cventry{}{}{NGO Observatorio de las Violencias de Género - Ahora Que Si Nos Ven}{2016-Present}{}{}
{
It means "Gender Violence Observatory - Now that you see us". The organization was born out of the NiUnaMenos (not one woman less) movement against femicide in 2015 with the need to track the femicide in Argentina. Since Femicide is defined as the most extreme form of gender-based violence, leading to the death of women solely because of their gender,
we decided to start monitoring local news and use that as a base to track femicides.I have lead and participated in different activities:
}
\begin{itemize}
  \item Support and organize feminist protests to legalize/decriminalize abortion in Argentina.
  \item Coordinate and execute different campaigns to create awareness and make gender issues, specially feminides, more visible and understandable at all levels of society. For example, in 2019 we followed Elina Chauvel 's artistic installation where we showed numerous women's shoes, one for each victim we tracked during that year.
  \item Coordinate with different local organizations campaigns and actions for different dates like March 08, abortion protests, etc.
  \item Develop and maintain our website {\url{http://ahoraquesinosven.com.ar/}}
\newline
\end{itemize}

\cventry{}{}{NGO Scouts de Argentina}{1999-2012}{}{}
{Leader responsible for the planning of activities for children age from 11 to 14 years old related to camping and solidarity based activities in the local community.}


\section{Work Experience}
\cventry[1em]{06/2020--Present} {Senior Product Manager}{Global Fishing Watch}{Washington DC, United States - Working remotely - Part-time }{}{}{}
  Manage the product roadmap of core products and consolidate input from Engineering, Research, Analysis, Communications, and Transparency teams to co-develop, maintain and communicate product roadmap(s).
\\

As a Product Manager my main tasks and responsibilities are:
\begin{itemize}
  \item Manage delivery timelines and reinforce internal and external expectations for product availability, ensure timely execution, and alignment with the innovation roadmap.
  \item Track and report progress to project funders and stakeholders
  \item Internally reinforce product awareness and knowledge (training/awareness/QAs).
  \item Introduce standard systems for reinforcing product life cycle and interconnectedness of the organization.
  \item Engage with geographically diverse internal and external users to develop key use cases, user journeys, and requirements for products.
  \item Find interconnections between existing products and new user requirements. Maintain the collection of user needs and prioritize features with regional and engineering teams.
  \item Collaborate with internal teams to prepare external user research activities, run field studies, and communicate findings from the study to inform product requirements.
  \item Organize product-related workshops in coordination with the research team.
  \item Collaborate with communications teams to create intuitive documentation, user guides and demos for products, including FAQs and video tutorials. Coordinate product training with regional teams.
  \item Coordinate with Communications to develop product-related communication plan, key comms assets, and external messaging with funders and science partners.
\newline
\newline
\end{itemize}

\cventry[1em]{01/2018--06/2020} {Senior Product Manager}{QU - Beyond POS}{Bethesda MD, United States - Working remotely - Part-time }{}{}{}
I started working as Scrum Master to help implement Scrum methodology and change Dev process and culture. Currently working as Product and Data Analyst.
\\

As Product and Data Analyst my main tasks and responsibilities were:
\begin{itemize}
  \item Analyze marketplace to understand industry and competitive landscape
  \item Work with customers, customer experience, sales, marketing, and engineering teams to develop requirements for the next generation of our product
  \item Monitor Google Analytics and analyze data to improve the product
  \item Write, prioritize and manage a backlog of epics and stories for your area of focus
  \item Work with our UX team to develop user interaction diagrams and mockups
  \item Be the voice of the customer, understand different user personas, and conduct user acceptance
  \item Write documentation and release notes
  \item Communicate product features, benefits to customers, sales team and other stakeholders through presentations and webinars
\newline{}
\end{itemize}

\cventry[1em]{04/2014--01/2018} {Product \& Data Analyst}{YaSabe \& JobBot  }{Mclean VA,  United States - Working remotely}{}{}{}
Functional / Requirements Analysis and Business and Data Analysis in an agile methodology environment for a US Startup. Organize Planning Meetings and Product Backlog
\\

As  Data Analyst, my tasks and responsibilities were:
\begin{itemize}
  \item Leverage data to provide recommendations to resolve issues for specific business groups
  \item Monitor Google Analytics and other analytics tools on a daily basis
  \item Leveraging Google Analytics trends, goals, and other data - determine areas of improvement on the site and in the product and content
  \item Perform customer and user data analysis to look for trends in user data/usage
  \item Produce regular reports on website site traffic, site usage patterns, and impact of new features
  \item Create necessary documents and presentations to support business development, marketing and product presentations
\newline
 \end{itemize}


As part of  Functional / Requirements Analysis, my tasks and responsibilities were:
\begin{itemize}
  \item Work closely with the Product Manager to define and document product specifications and requirements
  \item Elicit requirements using a variety of techniques including surveys, interviews, facilitated sessions, focus groups
  \item Translate product specifications into technical requirements
  \item Work closely with the technical and QA teams to ensure all product requirements perform well
  \item Facilitate User Acceptance Testing with business users
\newline
\newline
\end{itemize}



\cventry[1em]{09/2011--04/2014}{Functional Analyst Team Leader \& Product Analyst of Concerto}{AXA Assistance Argentina}{Buenos Aires, Argentina - On site}{}{}{}

I started in AXA as a Functional Analyst and I was in charge of elicitation and creating the functional documents of my projects related to changes on the AXA internal product called "Concerto". I also was in charge of the knowledge transfer and follow up of the testers assigned to my projects. In January 2013, I was promoted to Team Leader and in my team there were 5 analysts
\\

As Functional Analyst Team Leader, my responsibilities were:
\begin{itemize}
  \item Functional responsible of "Concerto" (a .net application developed in Paris and Buenos Aires, it involves the full AXA business process from the call center to billing)
  \item Propose improvements and new functionalities in Concerto
  \item Assign the functional analysts of my team in the different projects based on the knowledge and availability
  \item Estimations
  \item Follow up of all analysts' projects in order to meet budget and milestones
  \item Work on the yearly goal of the analysts and in the performance review
  \item Organize training when a new release before a new release is launched
  \item Unblock any member of my team in order to meet deadlines
  \item Create status reports of the analysts' assignments
\newline{}
\newline
\end{itemize}


\cventry[1em]{02/2011--08/2011}{Senior Business Analyst \& Project Leader}{DIRECTV}{Buenos Aires, Argentina - Onsite}{}{}{}
Working in the Regional IT team and my responsibility is to understand the new processes in order to implement them in all Panamericana region from Directv.
\\

\cventry[1em]{10/2010--01/2011}{Senior Functional Analyst}{Freelancer}{Buenos Aires, Argentina - Onsite}{}{}{}
I worked for a customer from UK and I was in charge of all functional documentation and change requests of 2 e-commerce sites: Argos and CarpetRight from UK.


\begin{itemize}
  \item Update meetings
  \item Analyze customer’s needs in order to identify business requirements
  \item Create prototypes in order to validate proposals
  \item Follow up the development of my proposals and support to QA and Testers
\newline
\newline
\end{itemize}

\cventry[1em]{03/2010--09/2010}{Senior Functional Analyst}{Summit Media (Globant Contractor)}{Prague, Check Republic - Onsite}{}{}{}
I have worked as Functional Analyst abroad for 6 months (in Prague, Czech Republic) and I was in charge of the functional analyst team.
The development team was spread all over the world (India, Argentina, UK and Prague).

\begin{itemize}
  \item My responsibility was to elicit requirements and offer different solutions and approaches to solve the business need.
  \item Pre sales support: I had to create Functional Specification and I have participated in estimations
\newline
\newline
\end{itemize}

\cventry[1em]{08/2009--02/2010}{Senior Functional Analyst and Project Leader}{IntelligenX}{Buenos Aires, Argentina - Onsite}{}{}{}
Requirements elicitation, user interviews, project planning and follow up.

\begin{itemize}
  \item Methodology: Agile
  \item Writing user stories
  \item Document systems functionalities in APT format in the project
  \item Write user manuals and document business processes
\newline
\newline
\end{itemize}

\cventry[1em]{01/2008--07/2009}{Semi Senior Functional Analyst}{Globant}{Buenos Aires, Argentina - Onsite}{}{}{}
I was the only Functional Analyst in a team of 30 developers (20 in Buenos Aires, 5 in Tandil and 5 in US) and 5 QC Analysts.
The system we built was developed in Java and .Net using an object oriented design

\begin{itemize}
  \item Methodology: Agile
  \item Extracting requirements and documenting those definitions
  \item Wrote user stories for the projects: GLOW and DNA (both internal projects)
  \item Wrote user manuals
  \item GLOW and DNA Training
  \item Pre sales support: I had to create Functional Specification and I have helped in estimations
  \item In charge of the planning and development of a UML Training Course
  \newline
  \newline
\end{itemize}


\cventry[1em]{06/2005--12/2007}{Java and .Net Developer, Functional Analyst SSr and Project Leader}{Patagonia Technologies}{Buenos Aires, Argentina - Onsite}{}{}{}

\begin{itemize}
  \item Gathering requirements and Functional analysis
  \item Use Case specifications and design of sequence and activity diagrams
  \item User Interface prototype specification and development
  \item Estimating resources
  \item Follow up meeting and minutes
  \item Development in Java and .net framework
  \newline
\end{itemize}

Projects in Patagonia:
\begin{itemize}
  \item 2006/2007 – Telecom: \newline{}
I was involved in four projects: Gestión de Emprendimientos, Portal de Autogestión, Estadísticas 0800 y Mi Cuenta.
My responsibilities as Functional Analyst are extracting requirements, defining and documenting those business requirements; design use cases using UML;  user Interface prototype specification. Also, as a Project Leader I am responsible for monitoring progress in all project simultaneously.

  \item 2006 –  La Caja de Ahorro y Seguro \newline{}
I was involved in one project: HDLC
My main responsibility was to develop in Java an internal tool for the company. (Eclipse 3.1, Hibernate, Struts)

  \item 2005 – Telecom \newline{}
I was involved in one project: Comisiones
My responsibilities were to extract the requirements, research and prioritize user requirements; Use Case specification and User Interface prototype specification and development.

  \item 2005 – INSA \newline{}
I was involved in one project: Gestor de Residuos
My responsibilities were to analyze user needs to determine functional requirements, functional analysis and design use cases and sequence diagrams.

  \item 2005 – Solis System \newline{}
I was involved in one project: NEP
My responsibilities were to analyze requirements, use case specification and design of activity and sequence diagrams.
Development in Java (win application using Netbeans).
Development in Visual Studio 2003 .NET
All communication were in English
\newline{}
\end{itemize}


\cventry[1em]{03/2004--05/2005}{Data Center analyst}{EDS}{Buenos Aires, Argentina - Onsite}{}{}{}

\begin{itemize}
  \item Audit tape process (Data center)
  \item Monitoring bank processes (mainframe, TSO (ispf))
  \item Coordination of technical services
  \item Inventory of modems and crypt devices
  \newline{}
\end{itemize}

\cventry[1em]{06/2003--02/2004}{Administrative Analyst}{LatinGrafica}{Buenos Aires, Argentina - Onsite}{}{}{}
\begin{itemize}
  \item Calculate budget (magazines, books, etc)
  \item Negotiate with suppliers
\newline{}
\end{itemize}

\section{Skills}

\cvitem{Senior Level}{Project Management, Define products strategy,  Capacity building skills,  Product definition, User activity tracking data analysis, Requirements elicitation, creating use cases/user
  Stories/product catalog, Estimations, Project scheduling, SQL, HTML, JSON, UML, Agile
  methodologies, User experience, Usability, Ecommerce, User training}

\cvitem{Semi-senior Level}{.NET framework, Java, MS SQL Server, MySQL,  Web Servers, REST, JSON, XML, SOA, Drupal, PMBOK Guide (PMI)}

\cvitem{Junior Level}{JavaScript,Ruby, Nodejs, JSP / Servlet / JavaBeans, Linux,  Eclipse, Netbeans, Hibernate, Struts, Spring, Nhibernate,
  Spring.Net, Tomcat,Prolog, Smalltalk}



\subsection{Application and tools}

\begin{itemize}
  \item Tools for modeling and documenting software systems: Advance level in modeling using UML in different tools like Enterprise Architect, Rational, Power AMC, TFS, etc
  \item Tools for issue tracking, task tracking, and project management: JIRA, Bugzilla, Quality Center, TFS, Target Process, Trello, Yadiz, etc.
  \item Tools for sharing information: Confluence, Sharepoint, internal wikis, Google Wiki.

\end{itemize}

\clearpage
\end{document}
