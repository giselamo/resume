\documentclass[11pt,a4paper,sans]{moderncv}

% style options are 'casual' (default), 'classic', 'oldstyle' and 'banking'
\moderncvstyle{banking}
% color options 'purple' (default), 'orange', 'green', 'red', 'purple', 'grey' and  'black'
\moderncvcolor{purple}

% character encoding
\usepackage[utf8]{inputenc}

% adjust the page margins
\usepackage[scale=0.8]{geometry}

% personal data
\name{GISELA}{MORINIGO}
\address{Monroe 3772 1F}{Ciudad de Buenos Aires (CP 1430)}{Bs. As., Argentina}
\phone[mobile]{54-11-5972-0764}
\email{gisemo@gmail.com}
\extrainfo{Pasaporte 29506398N, US VISA B1}

%----------------------------------------------------------------------------------
%            content
%----------------------------------------------------------------------------------
\begin{document}

%-----       resume       ---------------------------------------------------------
\makecvtitle

\section{Educación}

\cventry{2020--Al Presente}{Licenciatura en Antropología Social y Cultural}{Universidad Nacional de San Martín, UNSAM}{Grado}{}{}
\cventry{2012--2014}{Especialización en Ingenieria de Software}{Universidad Católica de Buenos Aires, FCFI}{Post-grado}{}{}
\cventry{2000--2006}{Ingeniera en Sistemas de Información}{Universidad Tecnológica Nacional, FRBA}{Grado}{}{}


\section{Objetivos personales}

Mi profesión como Ingeniera en Sistemas de Información me llevó a tomar diversos roles en el mundo del desarrollo de software, tanto en la industria como en organizaciones no gubernamentales. En los últimos 6 años, he estado participando activamente en el movimiento feminista argentino. Me sumé a una organización local sin fines de lucro llamada "Ahora Que Si Nos Ven". Es un observatorio de las violencias de género que tiene como objetivo visibilizar y denunciar las violencias machistas que afectan la libertad de las mujeres, trans, travestis, lesbianas y personas no binaries.
\newline
Mi objetivo es plasmar mi experiencia y conocimientos en la administración de proyectos, comunicación, habilidades interpersonales y conocimientos técnicos en una organización relacionada con género o en una organización social.
Disfruto mucho trabajar con equipos de diferentes culturas y paises. Soy una persona muy analítica, con buenas habilidades de comunicación y resolución de problemas. Me gustan los entornos dinámicos y me adapto fácilmente a los cambios.

\section{Idiomas}

\cventry{}{}{Español}{Nativo}{}{}
\cventry{}{Trabajo en estrecha colaboración con personas de habla inglesa}{Inglés}{Fluido}{}{\break}
\cventry{}{Trabajé con personas de habla francesa.}{Francés}{Intermediario}{}{\break}
\cventry{}{}{Italiano}{Básico}{}{}

\section{Actividades e intereses de voluntariado}
\cventry{}{}{ONG Observatorio de las Violencias de Género - Ahora Que Si Nos Ven}{2016-Al Presente}{}{}
{
La organización surgió luego de la gran movilización del 3 de junio del 2015 en la que la sociedad entera exigió Ni Una Menos.  Relevamos los femicidios en la Argentina a partir del análisis de medios gráficos y digitales de todo el país. De igual manera realizamos monitoreo de leyes e investigaciones relativas a los distintos tipos de violencias que sufrimos las mujeres, trans, travestis, lesbianas y personas no binaries en todos los ámbitos en los que desarrollamos nuestras relaciones interpersonales: hogar, trabajo, espacio públicos, instituciones educativas, etc. Participo y lidero diferentes actividades:
}
\begin{itemize}
  \item Apoyé y ayudé en la organización de los "martes verdes", las protestas feministas para legalizar/despenalizar el aborto en Argentina.
  \item Coordiné y ejecuté diferentes campañas para crear conciencia y hacer que los temas de género, especialmente feminides, sean más visibles y comprensibles en todos los niveles de la sociedad. Por ejemplo, en 2019 seguimos la instalación artística de Elina Chauvel donde mostramos numerosos zapatos de mujer, uno por cada víctima que rastreamos durante ese año.
  \item Ayudo en la coordinación con diferentes organizaciones locales campañas y acciones para diferentes fechas como el 08 de marzo, protestas por el aborto, etc.
  \item Desarrollé y mantengo el sitio web de nuestra organización {\url{http://ahoraquesinosven.com.ar/}}
\newline
\end{itemize}

\cventry{}{}{ONG Scouts de Argentina}{1999-2012}{}{}
{Líder responsable de la planificación de actividades para niños de 11 a 14 años relacionadas con campamentos y actividades solidarias en la comunidad local.}


\section{Experiencia laboral}
\cventry[1em]{06/2020--Al Presente} {Senior Product Manager}{Global Fishing Watch}{Washington DC, Estados Unidos - Trabajo remoto - Part-time }{}{}{}
  Administro el roadmap los productos principales, relevo requerimientos de diferentes actores como los equipos de Ingeniería, Investigación, Análisis, Comunicaciones y Transparencia para desarrollar conjuntamente el producto de software.
\newline
\newline

\cventry[1em]{01/2018--06/2020} {Senior Product Manager}{QU - Beyond POS}{Bethesda MD, Estados Unidos - Trabajo remoto - Part-time }{}{}{}
Empecé a trabajar como Scrum Master para ayudar a implementar metodologías ágiles, especificamente Scrum y cambiar el proceso y la cultura de equipo de desarrollo. Luego me desempeñe como Analista de Producto y Datos.
\newline
\newline

\cventry[1em]{04/2014--01/2018} {Product \& Data Analyst}{YaSabe \& JobBot  }{Mclean VA,  Estados Unidos - Trabajo remoto}{}{}{}
Análisis Funcional / de Requerimientos y Análisis de Negocios y Datos en un entorno de metodología ágil para una Startup de Estados Unidos. Organizar reuniones de planificación y gestión de los requerimientos de los productos.
\newline
\newline


\cventry[1em]{09/2011--04/2014}{Functional Analyst Team Leader \& Product Analyst of Concerto}{AXA Assistance Argentina}{Buenos Aires, Argentina - Presencial}{}{}{}
Empecé en AXA como Analista Funcional y estaba a cargo de la elicitación y creación de los documentos funcionales de mis proyectos relacionados con los cambios en el producto interno de AXA llamado "Concerto". También estuve a cargo de la transferencia de conocimiento y seguimiento de los testers asignados a mis proyectos. En enero de 2013 fui ascendido a Team Leader y en mi equipo había 5 analistas.
\newline
\newline

\cventry[1em]{02/2011--08/2011}{Senior Business Analyst \& Project Leader}{DIRECTV}{Buenos Aires, Argentina - Presencial}{}{}{}
Trabajo en el equipo Regional de Tecnologias de la Información y mi responsabilidad era entender los nuevos procesos para implementarlos en toda la región Panamericana de Directv.
\newline

\cventry[1em]{10/2010--01/2011}{Senior Functional Analyst}{Freelancer}{Buenos Aires, Argentina - Presencial}{}{}{}
Trabajé para un cliente del Reino Unido y estaba a cargo de toda la documentación funcional y las solicitudes de cambio de 2 sitios de comercio electrónico: Argos y CarpetRight del Reino Unido.
\newline
\newline

\cventry[1em]{03/2010--09/2010}{Senior Functional Analyst}{Summit Media (Globant Contractor)}{Praga, República Checa - Presencial}{}{}{}
Trabajé como Analista Funcional en el extranjero durante 6 meses (en Praga, República Checa) y estuve a cargo del equipo de analistas funcionales. El equipo de desarrollo estaba repartido por todo el mundo (India, Argentina, Reino Unido y Praga).
\newline
\newline

\cventry[1em]{08/2009--02/2010}{Senior Functional Analyst and Project Leader}{IntelligenX}{Buenos Aires, Argentina - Presencial}{}{}{}
Elicitación de requisitos, entrevistas a usuarios, planificación y seguimiento de proyectos.
\newline
\newline

\cventry[1em]{01/2008--07/2009}{Semi Senior Functional Analyst}{Globant}{Buenos Aires, Argentina - Presencial}{}{}{}
Fui la única Analista Funcional en un equipo de 30 desarrolladores (20 en Buenos Aires, 5 en Tandil y 5 en USA) y 5 Analistas de QC.
El sistema que construimos fue desarrollado en Java y .Net usando un diseño orientado a objetos.
\newline
\newline


\cventry[1em]{06/2005--12/2007}{Java and .Net Developer, Functional Analyst SSr and Project Leader}{Patagonia Technologies}{Buenos Aires, Argentina - Presencial}{}{}{}
Desarrollé en Java y .Net distintos sistemas. También tuve un rol de analista funcional, engargandome de elicitar requisitos y definir el alcance de los proyectos.
\newline
\newline


\cventry[1em]{03/2004--05/2005}{Data Center analyst}{EDS}{Buenos Aires, Argentina - Presencial}{}{}{}
Auditoria de cintas. Control de procesamiento de carga de bancos (entorno mainframe, utilización de TSO (ispf)). Asistencia en coordinación de las rutas de  los técnicos. Inventario de encriptores y módems
\newline
\newline

\cventry[1em]{06/2003--02/2004}{Administrative Analyst}{LatinGrafica}{Buenos Aires, Argentina - Presencial}{}{}{}
Cálculo de presupuestos y trato con proveedores.
\newline
\newline

\section{Skills}

\cvitem{Nivel Senior}{ Administración de proyectos, definición de estrategia de productos de software, relevamiento de requerimientos, crear casos de usos, documentos funcionales, user stories, seguimiento de proyectos  a un nivel de Project Leader, SQL, UML, Metodologías Agiles, Estimaciones, Planificación, MS Project, Usabilidad, User experience, User guides.}

\cvitem{Nivel Semi-senior}{.NET framework, Java, MS SQL Server, MySQL,  Web Servers, REST, JSON, XML, SOA, Drupal, PMBOK Guide (PMI)}

\cvitem{Nivel Junior}{JavaScript,Ruby, Nodejs, JSP / Servlet / JavaBeans, Linux,  Eclipse, Netbeans, Hibernate, Struts, Spring, Nhibernate, Spring.Net, Tomcat,Prolog, Smalltalk}


\subsection{Aplicaciones y herramientas}

\begin{itemize}
  \item Herramientas para modelar y documentar sistemas de software: Enterprise Architect, Rational, Power AMC, TFS, etc
  \item Herramientas para el seguimiento de tareas y la gestión de proyectos: JIRA, Trello, Bugzilla, Quality Center, TFS, Target Process, Trello, Yadiz, etc.
  \item Herramientas para compartir información: Confluence, Google Drive, Sharepoint, internal wikis, Google Wiki.

\end{itemize}

\clearpage
\end{document}
